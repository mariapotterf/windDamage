\documentclass[]{elsarticle} %review=doublespace preprint=single 5p=2 column
%%% Begin My package additions %%%%%%%%%%%%%%%%%%%
\usepackage[hyphens]{url}

  \journal{Forest Ecology and Management} % Sets Journal name


\usepackage{lineno} % add
\providecommand{\tightlist}{%
  \setlength{\itemsep}{0pt}\setlength{\parskip}{0pt}}

\usepackage{graphicx}
\usepackage{booktabs} % book-quality tables
%%%%%%%%%%%%%%%% end my additions to header

\usepackage[T1]{fontenc}
\usepackage{lmodern}
\usepackage{amssymb,amsmath}
\usepackage{ifxetex,ifluatex}
\usepackage{fixltx2e} % provides \textsubscript
% use upquote if available, for straight quotes in verbatim environments
\IfFileExists{upquote.sty}{\usepackage{upquote}}{}
\ifnum 0\ifxetex 1\fi\ifluatex 1\fi=0 % if pdftex
  \usepackage[utf8]{inputenc}
\else % if luatex or xelatex
  \usepackage{fontspec}
  \ifxetex
    \usepackage{xltxtra,xunicode}
  \fi
  \defaultfontfeatures{Mapping=tex-text,Scale=MatchLowercase}
  \newcommand{\euro}{€}
\fi
% use microtype if available
\IfFileExists{microtype.sty}{\usepackage{microtype}}{}
\bibliographystyle{elsarticle-harv}
\usepackage{longtable}
\usepackage{graphicx}
% We will generate all images so they have a width \maxwidth. This means
% that they will get their normal width if they fit onto the page, but
% are scaled down if they would overflow the margins.
\makeatletter
\def\maxwidth{\ifdim\Gin@nat@width>\linewidth\linewidth
\else\Gin@nat@width\fi}
\makeatother
\let\Oldincludegraphics\includegraphics
\renewcommand{\includegraphics}[1]{\Oldincludegraphics[width=\maxwidth]{#1}}
\ifxetex
  \usepackage[setpagesize=false, % page size defined by xetex
              unicode=false, % unicode breaks when used with xetex
              xetex]{hyperref}
\else
  \usepackage[unicode=true]{hyperref}
\fi
\hypersetup{breaklinks=true,
            bookmarks=true,
            pdfauthor={},
            pdftitle={Effects of forest management and harvest intensity on landscape level wind damage risk},
            colorlinks=false,
            urlcolor=blue,
            linkcolor=magenta,
            pdfborder={0 0 0}}
\urlstyle{same}  % don't use monospace font for urls

\setcounter{secnumdepth}{5}
% Pandoc toggle for numbering sections (defaults to be off)


% Pandoc header



\begin{document}
\begin{frontmatter}

  \title{Effects of forest management and harvest intensity on landscape level
wind damage risk}
    \author[Department of Biological and Environmental Science]{Mária Potterf\corref{1}}
   \ead{mpotterf@jyu.fi} 
    \author[Department of Biological and Environmental Science]{Kyle Eyvindson}
   \ead{kyle.j.eyvindson@jyu.fi} 
    \author[Department of Biological and Environmental Science]{Clemens Blattert}
   \ead{clemens.c.blattert@jyu.fi} 
    \author[Department of Biological and Environmental Science]{Daniel Burgas}
   \ead{daniel.d.burgas-riera@jyu.fi} 
    \author[Department of Biological and Environmental Science]{Mikko Mönkkönen}
   \ead{mikko.monkkonen@jyu.fi} 
      \address[University of Jyvaskyla]{Department of Biological and Environmental Science, University of
Jyvaskyla, P.O. Box 35, FI-40014 Jyvaskyla, Finland}
    \address[Wisdom]{This is wisdom address}
    \address[LUKE]{THIS is Luke address Department, Street, City, State, Zip}
      \cortext[1]{Corresponding Author}
    \cortext[]{}
  
  \begin{abstract}
  Future forest management needs to balance between sustainable timber,
  supporting non-woody ecosystem services and shelter biodiversity. Novel
  approaches as optimization of forest management to certain goals,
  increasing proportion of set-aside forest stands, or novel management
  approaches such as continuous forest cover emerge. However, novel ways
  of forest management shape forest stands structures over the landscape
  and will define the stand vulnerability to more frequent climatic
  disruptions, such as windthrows under hardening climate change. To
  understand how will the traditional (rotation forestry, RF) and novel
  forest managements techniques (continuous cover forest, CCF) combined
  using optimal management alternate the risk of wind damages over the
  harvest intensity gradient (ranging from completely setaside to highest
  harvesting rates), we combined the forest growth simulator under ranges
  of management regimes (RF, CCF and combined: ALL), optimized over the
  range of harvesting levels to calculate stand and landscape level wind
  damage risk for alternative paths of the forest management and
  harvesting levles over 100 years. We found that higher harvest intensity
  in RF lowers wind risk, whereas the wind risk increased under CCF and
  ALL scenarios. More intensive harvesting using RF produced more pulp,
  while CCF provided higher volumes of the standing and harvested log
  wood, which likely explains higher values of wind risk probability. RF
  slightly increased the number of stands with open edge, which remained
  stable under CCF and ALL regimes. Intensive harvesting may change
  species composition to favour Norway spruce which will further increase
  probability of wind damage in the future. Therefore, we suggest that
  forest managers should consider the target forest species composition to
  mitigate wind risk if aimed to support production of log wood over the
  pulp.
  \end{abstract}
  
 \end{frontmatter}

\newpage

\section{Introduction}\label{introduction}

Future forest ecosystems need to simultaneously provide timber,
non-woody ecosystem services such as carbon storage or human recreation,
and shelter biodiversity. Adaptive forest management strives to balance
between timber production, provision of non-woody ecosystem services,
and biodiversity. Existance of biodiversity, mostly attached to
existence of deadwood, is limited by harvesting intensity. Intensive
logging activities fragment forested landscapes. To balance between
biodiversity and timber economic gains, the proportion of set-aside
forests within commercial forests emerges, and new forest management
approaches are explored, such as continuous forest cover (Eyvindson et
al., 2021) and traditional harvesting regimes are becoming controversial
or requested to ban (\ldots{}). The increase of the set aside forests
within the commercial forests, as well as development of the new
management techniques affect landscape level structural diversity,
timing of the thinning, presence of absence of the final cuts in
rotation forestry or development of the larger trees within continuous
cover forestry or in set-aside forestry. The fundamental is the carefull
landscape level planning of the management actions balancing between
set-aside (unmanaged forests), intensive management and continuously
present forest cover.

Optimal management scenarios fullfil the specific objectives of the
society of forest owners to provide certain timber value, improve
provision of timber and non-timber ecosystem services, or improve
overall forest multifunctionality of the landscape. As such,
optimization provides the combination of the specific forest management
regimes on stand level. Althought the optimization process does not
necessary involve the spatial configuration of the stands, it
specifically assigns the particular regimes to individual stands and
therefore allows to recreate alternative dynamics landscapes shaped by
forest managements aggregated by optimal scenarios. As such, the spatial
configuration of the management regimes allows to estimate the
subsequent characteristics such as landscape level risk of wind damage.

The risk of the wind damage increases with current climate change and it
creates the major risk to the stability of the forest production.
Windthrows are unpredictable climatic disruption that shapes forest
structure and composition, and if left unsalvaged could create
opportiunities for deadwood dependent species and support local
biodiversity. From economical point of view, however, windthrows
massively abrupt the continuity of the timber supply, lowers timber
quality from log to pulp, increases the prices of unplanned salvage
harvesting (REF). To lower the risk of wind risk damage, current
suggestions include shortening the rotation period, promoting/avoiding
the wind resistant vs.~wind prone tree specuies, advocate for shortening
of the minimal stand age (Latvia REF). This however poses further
pressure on the multifucntionnal and multiple objective oriented
landscapes, which will provide habitats for endangered species, support
non-timber services and forest recreational use.

Traditional forest management regimes specialized in promoting timber
revenues while minimizing costs. In Fennoscandia, over the decades, the
traditional rotation forestry with multiple thinnings and final cuts
that over just mutlipel decades (from 1950) homogenized stands
structureal diversity, homogenized landscapes and increased forest
fragmenetations. On the other hand, forest management supporting
multifunctional landscapes, and promoting non-timber ecosystem services
requires implementation of the diverse set of management regimes
(Mönkkönen et al., 2014; Triviño et al., 2017). Furthermore, provision
of the endangered species habitats and non-woody ecosystem services are
provided on different scales where the planning scale should match or
overcome the scale that provided ecosystem (Pohjanmies et al., 2019).

Here we explore how the restriction of forest management practices,
along with the increasing level of harvest levels over the landscape
affects landscape level damage of wind risk and how much timber value is
put on risk under alternative regimes and extraction levels. Our study
for the first time evaluates the landcspeca level wind risk combined
with the forest growth simulator and long-term consequences of he
applied forest management practices. Therefore, we first calculate the
stand level wind risk over alternative landscapes and further explore
the likely drivers of the wind risk levels. We investigated how
restriction of forest management regimes combined with levels of
intensity of timber extraction will affects landscape level wind risk.

Our study aims to understand how types of forest management and
harvesting intensity affect stand and landcape level wind risk over
time, and how much timber volume is at the risk at time. We hypothetized
that RF will increase stand level wind risk due accumulated timber
volume before the final cut, and increase wind risk in landscape due to
increasing number of stands with open edges after the final cuts. On the
other hand, the CCF will likely homogenize the amount of timber volume
over landscape; therefore at any given time, the exposed timber volume
to wind damage will be lower in CCF compared to RF management. We wil
consider the pulp and log volume of the standing volume. We futher
investigate the dynamics of simulated stand characteristics as stand
dominant height, dominant tree speciecs by stand, age, number of trees
by stand, frequency of opend edge stands and thinning frequency over the
landscape over the harvest intensity gradient and time. We further
differentiate between the dynamics of the volume and wind risk in
managed and set-aside forests.

\section{Methods}\label{methods}

\subsection{Study area}\label{study-area}

Our study area represents a typical Finnish production forested
landscape with relatively structurally homogenous forests stands. In
total, we used 1475 forests stands aggregated within a single watershed
(number 14.534) in Central Finland, covering 2242 ha (Fig.
\ref{study_area}). Initial stand conditions were collected as open
source data from the Finnish Forest Centre (available on www.metsaan.fi)
providing currents stand conditions in 2016.

\begin{figure}
\centering
\includegraphics{/MyTemp/myGitLab/windDamage/externalFigs/studyArea_crop.png}
\caption{The study area located in Central Finland (watershed 14.534)
comprising 1475 forest stands.\label{study_area}}
\end{figure}

Our input dataset includes initial stand conditions (2016), alternative
forest growth under the range of forest managemnet regimes over 100
years and a set of optimal solutions balancing between intensifying
harvest levels and multifunctionnality (from completely set-aside,
i.e.~no management to maximal harvest gains). From the set of optimal
solutions we have calculated stand-level probability of wind damage (for
full details, please refer to Eyvindson et al. (2021) study. Fig.
\ref{workflow}) represents study workflow.

\begin{figure}
\centering
\includegraphics{/MyTemp/myGitLab/windDamage/externalFigs/overview2_horizontal.png}
\caption{The study workflow from collecting initial stand conditions
(2016) throught forest simulation growth under various ranges of forest
management, and constriuction of teh harvesting intensity gradient using
optimization to landscape level stand configurations.\label{workflow}}
\end{figure}

\subsection{Forest stand development under different
regimes}\label{forest-stand-development-under-different-regimes}

We simulated the development of the forests stands using SIMO forest
growth simulator ({\textbf{???}}) over 100 years, separated into 20
5-year sequences. Each stand could be managed by up to 58 different
management regimes (the total number of regimes per stands depended on
the initial stand conditions), including 17 regimes for rotation
forestry (RF), 40 variations of continuous cover forest (CCF) and one
set-aside (SA), where no management actions were taken. RF regimes
differed in timing of final felling, optional thinning (present/absent),
and increase in number of retained green trees after final cut (more
details in Eyvindson and Kangas (2018)). Basic CCF management follows
rules from Äijälä et al. (2014). To increase the range of CCF
managements, we varied two rules defining the timing of harvest: (i)
site-specific basal area and (ii) timing of the first thinning. We
modified the pre-defined site-specific basal area requirement (16m2/ha
for less fertile sites to 22m2/ha for fertile sites) prior to harvesting
by -3, ±0, +3, +6, and delayed the timing of the first harvest in 5 year
increments up to a delay of 45 years.

\subsection{Optimization}\label{optimization}

The optimal forest management explores the trade-offs between net
present income (NPI) and forest multifunctionality. NPI represents
economic value of the forests estimated by Metsähallitus (the Finnish
governmental organization managing state owned forests). Higher NPI
presents higher timber extraction and opposes the proportion of the
set-aside forest stands (i.e.~without active harvesting) over the
landscape. Optimization process over the NPI gradient was run using only
RF, only CCF management types, or all possible managements (RF and CCF,
further reffered as ALL) included over the gradient of NPI values, from
0 (representing completely set-aside or no management in all stands) to
maximal amount of extracted timber (leaving up to 5\% of SA stands). The
optimization balances between harvest intensity and landscape-level
multifunctionnality, including non-woody ecosystem services (climate
change mitigation), recreational activities and vertebrate and
non-vertebrate endangered species. The optimization resulted in 63
alternative collections of RF, CCF and ALL manamement regimes. We
converted the optimal solutions back to the development paths for every
stand under alternative management given group of management allowed
(CCF, RF, ALL) and levels of timber extraction (21). Each stand has only
one management regime by scenario. This allowed to reconstruct stand
structure on particular stand under given management regime at specific
time.

\subsection{Wind risk calculation}\label{wind-risk-calculation}

We have calculated the probability of wind damage based on Suvanto et
al. (2019) binomial generalized linear model with logit-link function
for each stand for every time step at each scenario. Suvanto et al.
(2019) model calculates the probability of the wind damage considering
available relevant open-access datasets including dominant tree species,
dominant tree height, time since thinning, predicted levels on maximal
wind speed, temperature sums, evaluated if stand has open edge, soil
type, mineral soil depth, site fertility and temperature sum (see
Suvanto et al. (2019) for all details). The final probability of wind
damage shows relative differences between stands, whereas the damage can
be only partial to the stand, but neglects the explicit spatial
locations of the future strongs winds. The parameters of the dominant
tree species, tree height, open edge and time since thinning were
dynamics under simulated management regimes. Parameters of maximal
predicted wind speed and temperature sums, as well as soil
characteristics, remained stable during our 100 years simulation. We
processed the datasets, calculated damage probability models, and
visualized results using R Development Core Team (2019).

\subsection{Data processing}\label{data-processing}

We calculated the probability of wind damage for each stand, scenario
and time intervals. Further, we averaged the wind risk values over the
scenarios to allow comparison between RF, CCF and ALL management regimes
groups over the harvesting gradient. We explored the mean wind risk (\%)
given the management groups and over harvest gradient (from set-aside to
maximal harvest levels). In addition, we investigated the mean levels of
the standing log and pulp timber volumes (m3) for scenarios, and total
sum of harvested pulp and log timber over simulation run (XXX). We
further investigated the changes in dynamics parameters (tree species,
dominant tree height, frequency of open edges and time since thinning)
over the harvest intensity gradient to understand how they contributed
to predicted wind risk values. As different tree species varied in wind
risk probability, we further groupped this variable by species to
understand differences between species, Specifically, we calculated mean
proportion of individual tree species (\%), mean tree heights (m),
proportion of the stands with open edge from total amount of stands by
species and mean thinning frequencies by year.

\section{Results}\label{results}

\subsection{Wind damage probability}\label{wind-damage-probability}

The level of wind risk remain the same with increasing harvest
intensity, while it is two time higher in CCF then in RF (Fig.3). In
set-aside stands is stabel until 5K EUR by ha, while afterwards
increases. This might be because of the selection of the less profitalne
stands as set-aside forests while increasing timber profits. Over time,
the probability of wind risk increases in CCF and RF management and
culminates towards the end of the simulation period, and in RF it
sharply decreases (FIG). the wind risdk in ciompletely set-aside forests
almost double over time. @ref\{fig\_meanRisk\_time\_NPI\}

\subsubsection{Timber volume at wind
risk}\label{timber-volume-at-wind-risk}

All three types of management groups responds similarly to set-aside
stands: stading timber volume decreases in set-aside stands, which
corresponds to preferential selection of the low quality stands to be
set as set-aside at the beginning of teh simulation period. Standing
mean log timber volume slightly increases with harvesting intensity
where increases from 50m3 to 75m3 with the highest increase in RF. In
set aside stands, the highest mean top stratum timber volume starts as
150 m3 and lowers to 50 m3/ha with increasing harvest intensity. In pulp
wood,RF has three times higher pulp volume compared to CCF as low
harvest intensities, the pulp volume in CCF remains relatively stable
with increasing harvest intensity.

All three types of management groups responds similarly to set-aside
stands: stading timber volume decreases in set-aside stands, which
corresponds to preferential selection of the low quality stands to be
set as set-aside at the beginning of teh simulation period. Standing
mean log timber volume slightly increases with harvesting intensity
where increases from 50m3 to 75m3 with the highest increase in RF. In
set aside stands, the highest mean top stratum timber volume starts as
150 m3 and lowers to 50 m3/ha with increasing harvest intensity. In pulp
wood,RF has three times higher pulp volume compared to CCF as low
harvest intensities, the pulp volume in CCF remains relatively stable
with increasing harvest intensity. Over time, the log volume culminates
in RF in year 2100 to up to 150 m3 in to top layers, while remaines
halved in CCF, whereas the pulp volume starts at the same levels for CCF
and RF in 2016 and in next 40 years (from 2060) in RF dominates the
proportion of teh pulp wood over the CCF. Overall, the log timber in in
SA stands increases over time, as they were not logged, and sligtthly
decreases in teh amount of pulp wood.

The intensification of the harvesting levels increases the proportion of
the pulp wood compared to log wood, especially in RF. In CCF, the
proportion among standing log and pulp volume remains at the same rate
(65:30) where the production of the log timber dominates. At the highest
intensity of timber extraction, pulp wood creates up to 50\% of teh
total standing volume See figure @ref(fig:proportion\_V\_pulp\_log) NEW
FIG @ref(proportion\_V\_pulp\_log).

In spruce dominated forests, the proportion of the log wood from top
stratum lowers by 50\% with increasing harvesting rate in all management
regimes applied. Interestingly, RF maintains the high production of the
pulp wood with increasing harvest intensity. For all species, the
proportion of the stand and top stratum volumes lowers by 47\% (spruce,
RF), while pulp lowers in CCF and ALL management (41) and remained the
same (mean 100 m3/ha)in RF with increasing harvesting rates.

\begin{figure}
\centering
\includegraphics{test_manus_files/figure-latex/fig_risk_V_all-1.pdf}
\caption{(\#fig:fig\_risk\_V\_all)Mean log (left) and pulp (right)
timber volume of the top tree layer in the stands over management groups
over harvesting intensity gradient (top) and over years (bottom)}
\end{figure}

\subsubsection{Changes in wind risk}\label{changes-in-wind-risk}

\subsection{Dynamic parameters contributing to wind
risk}\label{dynamic-parameters-contributing-to-wind-risk}

\subsubsection{Species composition}\label{species-composition}

The intensification of the harvesting changes the stand species
composition over time (Fig. \ref{species_change}). Intensification of
the harvesting favorize the proportion of the Norway spruce and others
(deciduous) tree species instead of Scots pine, which likely in turn
increases wind risk over the stands. Proportion of the pine lowers with
higher harvest intensities.

\begin{figure}
\centering
\includegraphics{test_manus_files/figure-latex/species_change-1.pdf}
\caption{(\#fig:species\_change)Changes in species composition under
different management groups and harvest intensity.}
\end{figure}

\subsubsection{Dominant tree heights}\label{dominant-tree-heights}

The intensification of the harvest reduce mean tree height for other
species to half (from 20 m to 10 m) in RF, while CCF slightly increases
mean tree heights.For pine and spruce trees, harvest intensification
maintain the mean tree height of the stand to 20 m(other) to 22 m
(spruce). Spruce is higher then pine and other stree species.

\begin{figure}
\centering
\includegraphics{test_manus_files/figure-latex/res_D_tree_height-1.pdf}
\caption{(\#fig:res\_D\_tree\_height)Mean dominant tree height under
different management groups and harvest intensity.}
\end{figure}

\includegraphics{test_manus_files/figure-latex/mean_stand_age-1.pdf}
\#\#\# Tree count

\includegraphics{test_manus_files/figure-latex/mean_tree_n-1.pdf}

\includegraphics{test_manus_files/figure-latex/wind_risk_SA-1.pdf}

\subsubsection{Frequency of open stands}\label{frequency-of-open-stands}

The proportion of the open edge stands remains very high over whole
simulation run, which is likely due to the fragmeneted landscape
(85-95\% of stands by species, Fig. 2, Fig. XX). However, increasing
harvest intensity lowers number of stands with open edge when dominated
by other species, whereas increases those numbers for pine and spruce
dominated spatnd, especially under RF. The CCF and ALL management groups
maintain the number of the open edge stands similar to completely set
aside stands.

\begin{figure}
\centering
\includegraphics{test_manus_files/figure-latex/fig_6_count_open_edge-1.pdf}
\caption{(\#fig:fig\_6\_count\_open\_edge)Yearly proportion stands with
open edge by species, management group and over the harvest intensity
gradient. Note y axis starts at 80.}
\end{figure}

\subsubsection{Thinning frequency}\label{thinning-frequency}

the highest frequency of thinnings is under CCF management regimes,
which lineraly increase with harvesting intensity for all tree species.
Numbers of thinnings in RF remains relatively low comparing to CCF and
ALL, and strongly increases at high harvest intensity (from 3K for pine
and from 5K for spruce). The frequency of thinnings in CCF increases 3-4
times in CCF comparing to RF.

\includegraphics{test_manus_files/figure-latex/unnamed-chunk-2-1.pdf}

\section{Discussion}\label{discussion}

\subsubsection{Wind risk dynamics}\label{wind-risk-dynamics}

We found that intensification of the harvesting regimes have different
conseuqences of resulting wind risk: where CCF management increases wind
risk, and RF lowers wind risk with increasing intensity.

\subsubsection{Wind risk drivers}\label{wind-risk-drivers}

\subsubsection{Future questions to
answer}\label{future-questions-to-answer}

\subsubsection{Recommendation for future
management}\label{recommendation-for-future-management}

\begin{itemize}
\tightlist
\item
  SA vs.~intensification of management
\item
  management types: CCF, RF or ALL or something else???
\end{itemize}

Wind (10 years return level of max wind speed REF) are estimated the
same over 100 years as well as temperature sums. How does could affects
the results?

In spite of inherent stochasticity of the wind and damage phenomena at
all spatial scales can be successfully modelled combining spatial
spatial datasets and ground earth observation data (Suvanto et al.
2019). Interpret Suvanto's map: there are 3 limitations: use values as
relative to each other - instead of exact probability valuesm, interpret
the map as relative differences in damage vulnerability Damage
probabilities do not refer to complete damage of the stand -- damage can
be only poartial, in some part of the stand (not spatially expicit) map
erepresent the forest vulnerability to the wind, but it is impossible to
predict the exact locations of future wind disturbances, given
uncertainities in future wind occurences

\begin{itemize}
\item
  the glm takes into account the dominant tree species but neglects the
  stand structure. WE neglected the stand structure in applying the
  raster level based glm model to stand level information
\item
  difficult to link the predicted wind risk to explicit wind damage
  volume as windthrows are unpredictanvle events in time and explicit
  locations.
\item
  We need to understand what risks about current management decisions
  involves and if they bring another challenges, such as increasing risk
  of wind damage. On the other hand, the set-aside stands, if protected,
  teh wind damage there should not be regarded as lost timber volume due
  to windthrow, as was not supposed to be logged anyway. Howvere, if
  windthrow happen in SA stands, the risk of subsequent disturbances
  such as \emph{Ips typographus} might increase the cost of the
  effective stand protection in following years.
\item
  SA forests have clearly higher wind risk than intensively managemed
  forests under RF. However, RF as highly specialized in provision of
  teh single benefit - timber - while deterioration non-woody ecosystem
  servoices and destroyng habitats for endangered species should be
  largely replaced over the landscapes by managements fulfillings
  multiple benefits Eyvindson et al. (2021).
\item
\end{itemize}

The shortening of the rotation period in Norway spruce forets might
slightly lower the wind and subsequent bark beetle distudrabnces but has
limited efficiencies and decreses under climate change (Zimová et al.,
2020)

\section{Conclusion}\label{conclusion}

\newpage

\section*{References}\label{references}
\addcontentsline{toc}{section}{References}

\hypertarget{refs}{}
\hypertarget{ref-Aijala2014a}{}
Äijälä, O., Koistinen, A., Sved, J., Vanhatalo, K., Väisänen, P., 2014.
Metsänhoidon suositukset {[}Good forest management recommendations{]}.
Forestry Development Center Tapio.

\hypertarget{ref-Eyvindson2020}{}
Eyvindson, K., Duflot, R., Triviño, M., Blattert, C., Potterf, M.,
Mönkkönen, M., 2021. High boreal forest multifunctionality requires
continuous cover forestry as a dominant management. Land Use Policy 100,
1--10.
doi:\href{https://doi.org/10.1016/j.landusepol.2020.104918}{10.1016/j.landusepol.2020.104918}

\hypertarget{ref-Eyvindson2018}{}
Eyvindson, K., Kangas, A., 2018. Guidelines for risk management in
forest planning --- what is risk and when is risk management useful?
Canadian Journal of Forest Research 48, 309--316.
doi:\href{https://doi.org/10.1139/cjfr-2017-0251}{10.1139/cjfr-2017-0251}

\hypertarget{ref-RDevelopmentCoreTeam2019}{}
R Development Core Team, 2019. R: A language and environment for
statistical computing.

\hypertarget{ref-Suvanto2019}{}
Suvanto, S., Peltoniemi, M., Tuominen, S., Strandström, M., Lehtonen,
A., 2019. High-resolution mapping of forest vulnerability to wind for
disturbance-aware forestry. Forest Ecology and Management 453, 117619.
doi:\href{https://doi.org/10.1016/j.foreco.2019.117619}{10.1016/j.foreco.2019.117619}

\hypertarget{ref-Zimova2020}{}
Zimová, S., Dobor, L., Hlásny, T., Rammer, W., Seidl, R., 2020. Reducing
rotation age to address increasing disturbances in Central Europe :
Potential and limitations. Forest Ecology and Management 1--50.


\end{document}


